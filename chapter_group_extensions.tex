\documentclass{article}
\usepackage[utf8]{inputenc}
\usepackage{amscd}
\usepackage{amsmath}
\usepackage{tikz-cd}
\usepackage[all]{xy}
\usepackage{amssymb,latexsym}
\usepackage{amsbsy}
\usepackage{longtable}
\usepackage{listings}
\usepackage{rotating}
\usepackage{times}
\usepackage{mathrsfs}
\usepackage{makeidx}
\usepackage{graphicx}
\usepackage{hyperref}
\usepackage{amsthm}
\newtheorem{proposition}{Proposition}[section]
\newtheorem{definition}{Definition}[section]
\newtheorem{lemma}{Lemma}[section]
\newtheorem{corollary}{Corollary}[section]
\theoremstyle{break} 
\newtheorem{theorem}{Theorem}[section]
\newtheorem{ex}{Example}[section]
\newtheorem{addition}{Addition}[section]
\newtheorem{rem}{Remark}[section]

\title{Computations in Galois cohomology}
\author{Aslam Ali }
\date{\today}
\def\core{\mathop{\mathrm{core}}\nolimits}
\def\cori{\mathop{\mathrm{cor}}\nolimits}
\def\inv{\mathop{\mathrm{inv}}\nolimits}
\def\Gal{\mathop{\mathrm{Gal}}\nolimits}
\def\N{\mathop{\mathrm{N}}\nolimits}
\def\incl{\mathop{\mathrm{incl}}\nolimits}
\def\Log{\mathop{\mathrm{Log}}\nolimits}	
\def\inva{\mathop{\mathrm{inv}}\nolimits}	
\def\Gal{\mathop{\mathrm{Gal}}\nolimits}
\def\id{\mathop{\mathrm{Id}}\nolimits}
\def\Hom{\mathop{\mathrm{Hom}}\nolimits}
\def\Cite{\mathop{\mathrm{Cite}}\nolimits}
\def\im{\mathop{\mathrm{Im}}\nolimits}
\def\ker{\mathop{\mathrm{ker}}\nolimits}
\def\rest{\mathop{\mathrm{res}}\nolimits}
\def\cori{\mathop{\mathrm{cor}}\nolimits}
\def\tor{\mathop{\mathrm{tor}}\nolimits}
\def\inf{\mathop{\mathrm{inf}}\nolimits}
\def\nr{\mathop{\mathrm{nr}}\nolimits}
\def\inv{\mathop{\mathrm{inv}}\nolimits}
\def\rad{\mathop{\mathrm{rad}}\nolimits}
\def\Irr{\mathop{\mathrm{Irr}}\nolimits}
\def\Aut{\mathop{\mathrm{Aut}}\nolimits}
\def\Det{\mathop{\mathrm{Det}}\nolimits}
\def\modulo{\mathop{\mathrm{mod}}\nolimits}
\def\ind{\mathop{\mathrm{ind}}\nolimits}
\def\det{\mathop{\mathrm{det}}\nolimits}
\def\Exp{\mathop{\mathrm{Exp}}\nolimits}
\def\Frob{\mathop{\mathrm{Frob}}\nolimits}	
\def\Trace{\mathop{\mathrm{Trace}}\nolimits}	
\def\Norm{\mathop{\mathrm{Norm}}\nolimits}	
\def\Gal{\mathop{\mathrm{Gal}}\nolimits}
\def\T{\mathop{\mathrm{T}}\nolimits}
\def\GCD{\mathop{\mathrm{GCD}}\nolimits}
\def\modulo{\mathop{\mathrm{mod }}\nolimits}
\def\valuation{\mathop{\mathrm{valuation}}\nolimits}
\def\v{\mathop{\mathrm{v}}\nolimits}
\def\Aut{\mathop{\mathrm{Aut}}\nolimits}
\def\N{\mathrm{N}}
\def\v{\mathrm{v}}
\def\m{\mathfrak{m}}
\begin{document}



\chapter{Applications of Global Fundamental Class} % Main chapter title

\label{Chapter5} % For referencing the chapter elsewhere, use \ref{Chapter1}

\lhead{Chapter 5. \emph{Applications}} % This is for the header on each page - perhaps a shortened title

%----------------------------------------------------------------------------------------

%\section{Applications}

\textbf{\huge{Sorry}$>>>>$To be written from the note$>>>$ This chapter is not finalised}


Let $E/L/K$ be a tower of Galois extension of number fields, then we have exact sequences
\[ 0 \rightarrow K^{*} \rightarrow J_K \rightarrow C_K,\]
where $J_K$ is id\`ele group and $C_K$ is id\`ele class group.
Since $H^{1}(L/K, C_L) =0$, we obtain cohomology long exact sequence  induced from above exact sequence as:
\[ 0 \rightarrow H^{2}(L/K, L^{*}) \rightarrow H^{2}(L/K ,J_L) \rightarrow H^{2}(L/K ,J_L) \rightarrow H^{3}(L/K , L^{*})\rightarrow \ldots .\]
We have similar results for the extension $E/K$ and $E/L$ and using these exact sequences, we obtain an exact commutative diagram:
\[\begin{tikzcd}
{ } & 0\arrow{d} & 0 \arrow{d} & 0\arrow{d} \\
0\arrow{r}  & H^{2}(L/K, L^{*} )\arrow{r}\arrow{d}{\inf} & H^{2}(L/K, J_{L} )\arrow{r}\arrow{d}{\inf} & H^{2}(L/K, C_{L} )\arrow{d}{\inf}\\
0\arrow{r}  & H^{2}(E/K, E^{*} )\arrow{r}\arrow{d}{\rest}  & H^{2}(E/K, J_{E} )\arrow{r}\arrow{d}{\rest} & H^{2}(E/K, C_{E} )\arrow{d}{\rest}\\   
0\arrow{r}  & H^{2}(E/L, E^{*} )\arrow{r} & H^{2}(E/L, J_{E} )\arrow{r}& H^{2}(E/L, C_{E} ).\\   
\end{tikzcd}
\]

Suppose $\overline{K}$ is the algebraic closure of $K$ and $E\rightarrow \overline{K}$ then we obtain a commutative diagram:
\[\begin{tikzcd}
{ } & 0\arrow{d} & 0 \arrow{d} & 0\arrow{d} \\
0\arrow{r}  & H^{2}(L/K, L^{*} )\arrow{r}{\gamma_1}\arrow{d}{\inf} & H^{2}(L/K, J_{L} )\arrow{r}{\epsilon_1}\arrow{d}{\inf} & H^{2}(L/K, C_{L} )\arrow{d}{\inf}\\
0\arrow{r}  & H^{2}(\overline{K}/K, \overline{K}^{*} )\arrow{r}{\gamma_2}\arrow{d}{\rest}  & H^{2}(\overline{K}/K, J_{\overline{K}} )\arrow{r}{\epsilon_2}\arrow{d}{\rest} & H^{2}(\overline{K}/K, C_{\overline{K}} )\arrow{d}{\rest}\\   
0\arrow{r}  & H^{2}(\overline{K}/L, \overline{K}^{*} )\arrow{r}{\gamma_3} & H^{2}(\overline{K}/L, J_{\overline{K}} )\arrow{r}{\epsilon_3}& H^{2}(\overline{K}/L, C_{\overline{K}} )\\   
\end{tikzcd}
\]
Define $\inv_{1} = \sum_{v}^{} \inv_v: H^{2}( L/K , J_L ) \rightarrow \mathbb{Q}/\mathbb{Z}$, then the sequence \[ \begin{tikzcd} 0\arrow{r}  & H^{2}(L/K, L^{*} )\arrow{r}{\gamma_1} & H^{2}(L/K, J_{L} )\arrow{r}{\inv_1} &  \mathbb{Q}/\mathbb{Z} \end{tikzcd} \]
is a complex form \cite{Cassels}.%, and the following sequence  \[ \begin{tikzcd} 0\arrow{r}  & H^{2}(L/K, L^{*} )\arrow{r}{\gamma_1} & H^{2}(L/K, J_{L} )\arrow{r}{\inv_{L/K}} &  \frac{1}{[L:K]}\mathbb{Z}/\mathbb{Z} .\end{tikzcd} \]


%We know $\inv_{v}( \inf(\alpha)) -= \inv_{v}(\alpha) $ for all $\alpha \in H^{2}(L/K , J_L)$, so we obtain another msp $\inv_{2}=\sum_{v}^{} \inv_v: H^{2}( \overline{K}/K , J_{\overline{K}} ) \rightarrow \mathbb{Q}/\mathbb{Z}$ such that the diagram
%\[
%\begin{tikzcd}
%H^{2}( L/K, J_L) \arrow{r}{inv_1} \arrow{d}{\inf} & \mathbb{Q}/\mathbb{Z}\arrow{d}{\id}\\
%H^{2}( \overline{K}/K , J_{\overline{K}} )\arrow{r}{\inv_2}& \mathbb{Q}/\mathbb{Z}
%\end{tikzcd}
%\]
By the axioms of class formation, we have $\inva_{w} (\rest ( \alpha))= [L_w : K_v] \inva_{v} (\alpha)$ where $\alpha \in H^{2}(\overline{K}/K , J_{\overline{K}} )$ and $w$ is a place of $L$ over $v$ of $K$. Let us define $\inv_{3}=\sum_{v}^{} \inv_v: H^{2}( \overline{K}/L , J_{\overline{K}} ) \rightarrow  \mathbb{Q}/\mathbb{Z}$, so we obtain another commutative diagram
\[
\begin{tikzcd}
H^{2}( \overline{K}/K , J_{\overline{K}} )\arrow{r}{\inva_2}\arrow{d}{\rest}& \mathbb{Q}/\mathbb{Z}\arrow{d}{[L:K]}\\
H^{2}( \overline{K}/L , J_{\overline{K}} )\arrow{r}{\inva_3}& \mathbb{Q}/\mathbb{Z}
\end{tikzcd}
\]
where for every place $v$ of $K$ we have $\sum_{w/v}^{}[L_w: K_v] = [L:K]$.
\begin{proposition}
	Let $L/K$ be a finite Galois field extensions with Galois group $G$, $v$ a place in $K$ and $w_0$ be a place of $L$ over $v$. Then there are mutually inverse isomorphisms
%	\[\begin{tikzcd}[column sep= large]
%	H^{r}( G, \prod_{w/v}^{} L_{w}^{\times} ) \arrow[transform canvas={xshift=0.7ex}] & \arrow[l, shift left =0.5 ex, "j_{w_{0}\cdot \rest}"] H^{r}( G_{w_{0}}, L_{w_{0}}^{\times} ).
%	\end{tikzcd}	
%	\] 

\[
\begin{tikzcd}[column sep= large]
	H^{r}( G, \prod_{w/v}^{} L_{w}^{\times} ) \arrow[shift right =-0.5 ex]{r}{\cori \cdot i_{w_{0}}} & \arrow[shift left = 0.5 ex]{l}{j_{w_{0}\cdot \rest}} H^{r}( G_{w_{0}}, L_{w_{0}}^{\times} ).
	\end{tikzcd}	
	\] 
Also for unit group $U_{w}$ of $L_w$ we have 
\[\begin{tikzcd}[column sep=large]
H^{r}( G, \prod_{w/v}^{} U_w ) \arrow[ shift left = 0.5ex]{r} {\cori \cdot i_{w_{0}}} & \arrow[ shift left =0.5 ex]{l} {j_{w_{0}\cdot \rest}} H^{r}( G_{w_{0}}, U_{w_{0}}^{\times} ).
\end{tikzcd}\]  

\end{proposition}
\begin{proof}
\cite[Proposition 7.2]{Cassels}.
\end{proof}

%\arrow{l}{ cores \cdot i_{w_{0}}}


\section{Ray Class Group}
\cite{Hofmann}.\\
As we have seen earlier the id\`ele class group of $K$ is $C_K= I_{K}/K^{\times}$. Let the notations $K$, $\mathcal{O}_{K}$ and $\mathfrak{p}_{K}$ be as of the Chapter \ref{Chapter4}. Then a modulus $\mathfrak{m}$ of $K$ is a formal product $\mathfrak{m}=\prod_{\mathfrak{p}}^{} \mathfrak{p}^{m_{p}}=\mathfrak{m}_{f}\cdot \mathfrak{m}_{\infty}$ where $m_p =0$ for almost all $\mathfrak{p}$,  $\v_{\mathfrak{p}}(\mathfrak{m}) \ge 0$ and $ \mathfrak{m}_{\infty}$, a set of real embeddings of $K$.
\begin{definition}
Let $K$ be a global field, $\mathfrak{m}$ be a modulus of $K$ and $I_{\mathfrak{m}}$ be the group of fractional ideals  prime to $\mathfrak{m}_f$ which is isomorphic to the free abelian group \[\bigoplus_{\mathfrak{p} \nmid \mathfrak{m}_f , \mathfrak{p}\hspace{2mm} finite } \mathbb{Z}\cdot \mathfrak{p}.  \]
Also, suppose 
\[ R_{\mathfrak{m}} = \{(a) \mid a \equiv 1 \hspace{2mm} \modulo  \mathfrak{m} \} \subset I_{\mathfrak{m}},\]
then the quotient $I_{\mathfrak{m}}/ R_{\mathfrak{m}}$ is known as the ray class group and denoted by $Cl_{\mathfrak{m}}.$ 
\end{definition}
In fact, when $\mathfrak{m} =1$ then $\mathfrak{m}_{\mathfrak{p}}=0$ for all $\mathfrak{p}$ then we obtain $Cl_{1}=I_{1}/R_{1}=Cl(\mathcal{O}_{K})$. An abelian field extension $K^{\mathfrak{m}}/ K$ associated to the ray class group $Cl_{\mathfrak{m}}(K)$ by class field theory is called the ray class field. In this case $\Gal(K^{\mathfrak{m}}/ K) \cong Cl_{\mathfrak{m}}(K)$. \\
Let  $\mathfrak{p}$ be a prime ideal of $K$ which is unramified in the abelian field extension $L/K$ and $\mathfrak{P}$ be the prime ideal of $L$ lying above $\mathfrak{p}$. Then $L_{\mathfrak{P}}/K_{\mathfrak{p}} $ is unramified local field extension. So, there exists a unique Frobenius automorphism $\Frob_{\mathfrak{p},L/K} \in \Gal(L/K)$ with $\Frob_{\mathfrak{p}, L/K}(x)\equiv x^q \mod \mathfrak{P}$ for all $x \in \mathcal{O}_{L}$, where $q$ is the number of elements in the residue class field of $K_{\mathfrak{p}}$. The decomposition group at prime ideal $\mathfrak{p}$ is $G_{\mathfrak{p}}= G(L_{\mathfrak{P}} / K_{\mathfrak{p}})= \langle \Frob_{\mathfrak{p}, L/K} \rangle\subset G(L/K)$. Let $\mathfrak{m}$ be a modulus  which is only divisible by ramified prime ideals of $L/K$, then we can define the Artin map $\psi_{L/K}: I_{\mathfrak{m}} \rightarrow \Gal(L/K)$ such that $\psi_{L/K}(\mathfrak{p})= \Frob_{\mathfrak{p}, L/K}$ for all $\mathfrak{p}$ not dividing $\mathfrak{m}_f$. \\
%\begin{theorem}
%Suppose $L/K$ be a finite normal extension of number fields then the set if primes in $K$ whic split completely/totally in $L$ has polar density $1/[L:K]$.
%\end{theorem}


\begin{theorem}
Let $L/K$ be a finite abelian extension of number fields and $\mathfrak{m}$ be a modulus for $K$ divisible by all ramified primes. Then the Artin map $\psi_{L/K}: I_{\mathfrak{m}} \rightarrow \Gal(L/K)$  is surjective.
\end{theorem}
\begin{proof}
Lecture notes page-21 pdf cite...
\end{proof}
From above theorem, we obtain an exact sequence:
\[1\rightarrow \ker(\psi_{L/K}) \rightarrow I_{\mathfrak{m}}\rightarrow \Gal(L/K) \rightarrow 1 .  \]
Thus, we find $R_{\mathfrak{m}} \subset \ker(\psi_{L/K})$ for a modulus $\mathfrak{m}$ of $K$. This shows that Artin map induces an isomorphism from quotient $ Cl_{\mathfrak{m}} = I_{\mathfrak{m}}/ R_{\mathfrak{m}}$ to $\Gal(L/K)$. But when $R_{\mathfrak{m}} = \ker(\psi_{L/K})$ then $Cl_{\mathfrak{m}}\simeq \Gal(L/K)$ and the field $L$ is the \textbf{ray class field} corresponding to modulus $\mathfrak{m}$.\\
Let $\mathfrak{p}$ be a prime ideal of $K$, $n\in \mathbb{Z}_{\geq 0}$ and define $U_{\mathfrak{p}}^{n}$ as
  \[
 U_{\mathfrak{p}}^{n}  = 
  \begin{cases}
 \mathcal{O}_{K_\mathfrak{p}}^{\times}, & \mathfrak{p} \text{ finite },  n =0,\\
 1+\pi_{\mathcal{O}_{K_\mathfrak{p}}}^{n}\mathcal{O}_{K_\mathfrak{p}} & \mathfrak{p} \text{ finite }, n > 0,\\
 K_{\mathfrak{p}}^{\times} & \mathfrak{p} \text{ real and } n =0 , \text{ or }  \mathfrak{p} \text{ complex},\\
  K_{\mathfrak{p}}^{\times,+}  & \mathfrak{p} \text{ real},  n>0.\\
 \end{cases}
 \]
% In fact $U_{\mathfrak{p}}^{n} \subset K_{\mathfrak{p}}^{\times}$ is open for all $\mathfrak{p}$
Let $\mathfrak{m} $ be a modulus  of $K$, then $U_{K, \mathfrak{m}} = \prod_{\mathfrak{p}}^{}U_{\mathfrak{p}}^{m_\mathfrak{p}}$ where $m_{\mathfrak{p}} = \v_{\mathfrak{p}}(\mathfrak{m})$, forms an open subgroup of $K$ because $U_{\mathfrak{p}}^{m_\mathfrak{p}} \subset K_{\mathfrak{p}}^{\times}$ is open for all $\mathfrak{p}$ and equals $\mathcal{O}_{K_\mathfrak{p}}^{\times}$ for almost all $\mathfrak{p}$.
\begin{proposition}
Any open subgroup of $J_{K}$ contains some $U_{K,\mathfrak{m}}$, and the quotient $|J_{K}/ K^{\times}U_{K,\mathfrak{m}}| <\infty$. 
\end{proposition}
\begin{proof}
Proposition 9.2, Note of M. Flach Ray Class group.
\end{proof}
Note: For a modulus $\mathfrak{m}$ of $K$ we have $Cl_{\mathfrak{m}}=J_{K}/ K^{\times}U_{K,\mathfrak{m}}.$\\
\begin{proposition}
Let $K$ be a number field and $\mathfrak{m}$ be a modulus of $K$. Suppose $\overline{U}_{K,\mathfrak{m}}$ is the image of $U_{K,\mathfrak{m}}$ in $C_K$, then we have
\[  C_K/\overline{U}_{K,\mathfrak{m}}= I_{K}/K^{\times}.U_{K,\mathfrak{m}} \xrightarrow{\sim} I_{\mathfrak{m}}/R_{\mathfrak{m}}=Cl_{\mathfrak{m}}  \]
and 
\[C_K/ N_{L/K}(C_L)= J_{K}/K^{\times}\cdot N_{L/K}(J_{L}^{\times}) \cong I_{K,\mathfrak{m}}/R_{\mathfrak{m}}\cdot N_{L/K}I_{L,\mathfrak{m}}  . \]
\end{proposition}
\begin{proof}
See note by M Flach , Prospostion 3.2.
\end{proof}
\begin{proposition}
Let $L/K$ be a normal extension of number fields and $\mathfrak{m}$ be a modulus of $L$ which is invariant under the $G=\Gal(L/K)$. Then for every subgroup $H$ of ray class group $Cl_{\mathfrak{m}}$ which is invariant under the action of $G$, there exists an abelian extension $E/L$ such that $E/K$ is normal. 
\end{proposition}
\begin{proof}
\cite{Hofmann}, Proposition $15$.
\end{proof}
Let $L/K$ be a normal extension of the fields. We can compute an abelian extension $E/L$ using \cite{Hofmann} so that $E/K$ is normal.


\begin{definition}
Let $L/K$ be a finite abelian field extension. A modulus $\mathfrak{m}$ of $K$ is said to be admissible for an abelian extension $L/K$ if and only if (almost) all primes $ \mathfrak{p} \in R_{\mathfrak{m}}$ of $K$ totally split in $L$.      \\
OR\\
There exist a modulus $\mathfrak{m}$ such that $\Gal(L/K)\cong Cl_{\mathfrak{m}}(K)$.
        
\end{definition}
Let $L/K$ be a finite Galois extension, $\Delta_{L/K}$ be the discriminant of the extension $L/K$ and $I_{K}(\Delta_{L/K})$ be the group of fractional $\mathbb{Z}_{K}$-ideals generated by the primes $\mathfrak{p}$ of $\mathcal{O}_{K}$ that do not divide $\Delta_{L/K}\mathcal{O}_{K}$. The Artin map for $L/K$ is defined as the homomorphism 
\[ \psi_{L/K}: I_{K}(\Delta_{L/K}) \rightarrow \Gal(L/K), \mathfrak{p}\mapsto \Frob_{\mathfrak{p}}.\]
  
\begin{theorem}[Kronecker-Weber]
Let $L/ \mathbb{Q}$ be an abelian extension then there exists an integer $m\in \mathbb{Z}_{\geq 0}$ such that the kernel of the Artin map $\psi_{L/\mathbb{Q}}$ consists of all $\mathbb{Z}-$ ideals $x\mathbb{Z}$ with $x>0$ and $x\equiv 1  \modulo  m$.
\end{theorem}

\begin{theorem}[Artin's Reciprocity]
Let $L/K$ be an abelian extension, then there is a nonzero ideal $\mathfrak{m}_{0} \mathbb{Z}_{K}$ such that the kernel of the Artin map defined as 
\[  \psi_{L/K}: I_{\mathfrak{m}_{0}} \rightarrow \Gal(L/K), \mathfrak{p}\mapsto \Frob_{\mathfrak{p}}\]
consists of all principal $\mathbb{Z}_{K}-$ ideals $x\mathbb{Z}_{K}$ with $x$ totally positive and $x\equiv 1 \mod \mathfrak{m}_{0}$.
\end{theorem}

%\begin{remark}
%Let $L/K$ be a finite abelian extension, $\mathfrak{m}=\mathfrak{m_0}\cdot \mathfrak{m_\infty}$ an admissible modulus for $K$ and $J^{\mathfrak{m}}$ be the group of fractional ideals coprime to $\mathfrak{m_f}$ then one can define Artin map as:
%\[ \psi_{L/K}:H^{\mathfrak{m}} \rightarrow \Gal(L/K),\hspace{3mm}  [\mathfrak{p}]\mapsto \Frob_{\mathfrak{\mathfrak{p}}}.\]
%\end{remark}
Let $P_{\mathfrak{m}}$ be the principal ideals in $I_{\mathfrak{m}}$ of $K$, then the ray group $R_{\mathfrak{m}}$ is contained in $P_{\mathfrak{m}}$ and from \cite{Cohen1} $I_{\mathfrak{m}}/P_{\mathfrak{m}}\cong Cl(K)$ for all modulus $\mathfrak{m}$. Because of the relations $R_{\mathfrak{m}}\subset P_{\mathfrak{m}} \subset I_{\mathfrak{m}}$, it is clear that $Cl_{\mathfrak{m}}$ is an extension of $Cl_{K}$ by a finite abelian group $P_{\mathfrak{m}}/R_{\mathfrak{m}}$. We have an exact sequence
\[\mathcal{O}^{\times}_{K} \rightarrow (\mathcal{O}_{K}/{\mathfrak{m}})^{\times} \rightarrow Cl_{\mathfrak{m}} \rightarrow Cl_{K} \rightarrow 0 . \]
where $ (\mathcal{O}_{K}/{\mathfrak{m}})^{\times} = (\mathcal{O}_{K}/{\mathfrak{m}_{0}})^{\times} \times \prod_{\mathfrak{p}\mid \mathfrak{m}_{\infty}}^{}\left\langle -1 \right\rangle$ and $x\in \mathcal{O}_{K}$ coprime to $\mathfrak{m}_{0}$ is mapped in the finite group $ (\mathcal{O}_{K}/{\mathfrak{m}_{0}})^{\times}$ as of its residue class modulo and the signs of its images under the real primes $\mathfrak{p}\mid \mathfrak{m}_{\infty}$. Then the quotient $(\mathcal{O}_{K}/{\mathfrak{m}_{0}})^{\times}/ \im[\mathcal{O}^{\times}_{K}]$  is isomorphic to $\Gal(H_{\mathfrak{m}}/H_{1}) $ given by Artin map where $H_{\mathfrak{m}}$ denotes the ray class field associated to modulus $\mathfrak{m}$.\\
\begin{lstlisting}
> x := PolynomialRing(Integers()).1;
> K := NumberField(x^2+5);
> O := MaximalOrder(K);
> m := 5*O;
> r,mr := RayClassGroup(m);    
> R,mR := RayResidueRing(m);
> U,mU := UnitGroup(o);
> f := hom<U->R|x:-> x@mU@@mR>;
> q,mq := quo<R|Image(f)>;
> Hm :=AbsoluteField(NumberField(RayClassField(m)));
> H1 :=AbsoluteField(HilbertClassField(K));
> IsSubfield(H1,Hm);
true Mapping from: FldNum: H1 to FldNum: Hm

\end{lstlisting}


Let $\mathfrak{m}$ be an admissible modulus for an abelian extension $L/K$, then the Artin map induces an isomorphism 
\[ \psi_{L/K}: Cl_{\mathfrak{m}} \rightarrow \Gal(L/K), \hspace{1cm} [\mathfrak{p}]\mapsto \Frob_{\mathfrak{p}}. \]
In fact, this map is surjective because of the triviality of extensions in which all primes totally split\cite{Cohen1}.\\
Let $\mathfrak{p}_{L}$ be the prime  of $L$ over the prime $\mathfrak{p}_{K}$ of $K$, then the order of $\Frob_{\mathfrak{p}}$  is $[\underline{L_{\mathfrak{p}_{L}}}:\underline{K_{\mathfrak{p}_{K}}}]=f_\mathfrak{p}$ where $\underline{K_{\mathfrak{p}}}$ and  $\underline{L_{\mathfrak{q}}}$ are the residue class fields of $K_{\mathfrak{p}_K}$  and $L_{\mathfrak{p}_L}$ respectively. For every prime ideal $\mathfrak{p}_{L}\in \mathcal{O}_{L}$ coprime to $\mathfrak{m}$, the norm $N_{L/K}(\mathfrak{p}_L)={\mathfrak{p}_K}^{f_{\mathfrak{p}}} $ is contained in the kernel of the Artin map.\\
Let $A_{\mathfrak{m}} \subset I_{\mathfrak{m}}$ be the ideal group which corresponds to $L$ so that $\ker(\psi_{L/K})= A_{\mathfrak{m}}/ P_{\mathfrak{m}}$ then $A_{\mathfrak{m}}= N_{L/K}(I_{\mathfrak{m}\mathbb{Z}_{L}})\cdot P_{\mathfrak{m}}$.
% because  $N_{L/K}(\mathfrak{q})= \mathfrak{p}^{|F_{K_{\mathfrak{p}}}|}$.
% fro every prime ideal $\mathfrak{q} \in \mathcal{O}_{L}$ coprime to $\mathfrak{m}$ i.

%\begin{theorem}
%Let $L/K$ be a finite Glaois field extension and $\mathfrak{m}$ be the $K$-modulus divisible by all the places of $K$ which ramify in $L$. Then the Artin map $\phi_{L/K,\mathfrak{m}} : J^{\mathfrak{m}} \rightarrow \Gal(L/K)$ is surjective and $N_{\mathfrak{m}}(L/K)$. When $\mathfrak{m}$ is admissible for $L/K$, $\ker(\phi_{L/K,\mathfrak{m}}) = P^{\mathfrak{m}}N_{\mathfrak{m}}(L/K)$ 
%\end{theorem}

Suppose $E/L/K$ is a tower of global field extensions such that $E/L$ is abelian and $L/K$ is normal. Let us denote by $\Gal(L/K)=G, \Gal(E/K)=\Sigma$ and $\Gal(E/L)=A$. So we have an exact sequence of Galois groups: 
\[ 1\rightarrow A \rightarrow \Sigma \rightarrow G \rightarrow 1. \]
In fact, $A\simeq C_{L}/N_{E/L}(C_E)$ by Artin isomorphism.\\



The following theorem is of the main interest to find information of $\Sigma$:
\begin{theorem}
Let $E/L/K$ be as above and so as Galois groups $A,\Sigma$ and $G$, then
\begin{enumerate}
\item Let $\gamma \in \Sigma$ have image $\overline{\gamma} \in G$. suppose $x \in C_{L}$ and $\psi: C_{L}\rightarrow A$ is the Artin map, then $\psi(\overline{\gamma}x)= \gamma \psi(x) \gamma^{-1}$.
\item Let $v \in H^{2}(G,A)$ be the class of the group extension $\Sigma$, $u_{L/K} \in H^2(G,C_L)$ be the fundamental class for $L/K$ and $\psi_{*}: H^2(G,C_L) \rightarrow H^2(G,A) $ is induced by the Artin map $\psi$ then $v= \psi_{*}(u_{L/K})$.
\end{enumerate}
\end{theorem}
\begin{proof}
\cite{Cassels}.
\end{proof}
We have seen earlier that for any Galois extension $L/K$ of global field we assume the finite approximation of $S$-id\`dele class group $C_{L,S}^{f}$ instead of $C_L$ and we compute the global fundamental class in $H^{2}(\Gal(L,K), C_{L,S}^{f})$. We choose a modulus $\mathfrak{m}$ for $L$ which is invariant under $G$ so that we obtain an abelian field extension $E/L$ corresponding to the ray class group $A=Cl_{\m}$. We define a map from $\psi_{*}:C_{L,S}^{f} \rightarrow A$
and then we compute  $v$ which is given by $\psi_{*}(u_{L/K}) $ using MAGMA. We summarize all in the following algorithm to construct a group extension for given an invariant modulus $\m$:


\textbf{Algorithm}\\
\textbf{Input} Given Galois extension $L/K$ with group $G=Gal(L/K)$ and a modulus $m \leq OL$ .\\%which corresponds a class field $A/L$.\\
\textbf{Output} Construct an abelian extension $A=Cl_{m}/U$ and compute the group extension $E$ which corresponds the Galois Group $A/K$.
\begin{enumerate}
\item Compute the global fundamental class $u\in H^{2}(G, C_{L,S}^{f}(L))$ for the extension $L/K$.
\item Compute an abelian extension $A/L$ which corresponds to the ray class group $Cl_{m}/U$ where $U$ is some unit group.% in MAGMA.
\item Compute a cohomology module $CM$ for an abelian extension $A$. 
\item Construct a map from finite approximation $S$-id\`ele Class group to cohomology module $CM$.
\item  Find a cocycle $\gamma$ in $H^{2}(G, CM)$ corresponding to $u$.
\item Return an extension of group
\[
\begin{tikzcd} A\arrow[hookrightarrow]{r}{\mu}&  E  \arrow[two heads]{r} & G  
\end{tikzcd}
\]
using $\gamma$ in MAGMA. 
\end{enumerate}



We have implemented the above algorithm for cyclic extensions of number fields and for Galois extensions of number fields which have an undecomposed prime. If a  normal extension  $L/K$ has no undecomposed prime then the global fundamental class has been computed using the compositum of a cyclic extension $L_{1}/K$ having degree $[L:K]$. In this case, the $S$-id\`ele class group $C_{L,S}^{f} $ has been computed with the fixed group of $S$-id\`ele class group  $C_{N,S}^{f}$, that is, 
\[ C_{L,S}^{f}= \left(C_{N,S}^{f}\right)^{\Gal(N/L)},\]
where $N$ is the compositum of the fields $L$ and $L_{1}$. Suppose the Galois group of $L/K$ is then the fundamental class $u$ of $L/K$ is a map
\[ u:G\times G \rightarrow C_{L,S}^{f}\]
which returns the values in $C_{L,S}^{f}$. we can implement the above algorithm one we are able to construct the map of step $(4)$ of the above algorithm. 

\begin{lstlisting}
> time CohL,f1CL,gfc, comp,req:=gfcUndecomposedcl(L,7);
Time: 0.510
> G:=Group(CohL);
> CL_L:=req[5];
> CL:=comp`CL;
> qCL:=comp`qCL;
> projJL:=req[2];
>  u := TwoCocycle(CohL,gfc);
> r,mr:=RayClassGroup(3^7*MaximalOrder(L));
> r;
Abelian Group isomorphic to Z/729
Defined on 1 generator
Relations:
    729*r.1 = 0
> A:= AbelianExtension(mr);
>   CM,w1,w2,w3 := CohomologyModule(A);
>   CohomologyGroup(CM, 2);
Full Quotient RSpace of degree 1 over Integer Ring
Column moduli:
[ 3 ]
> mm:= hom<car<G,G>->L| x:-> x@u@@f1CL@CL_L>;
> IdentifyTwoCocycle(CM,func<x | mm(x[1],x[2])@@mr@@w3>);
(1)
> Extension(CM,$1);
Finitely presented group on 2 generators
Relations
    $.2^729 = Id($)
    ($.1, $.2^-1) = Id($)
    $.1^6 * $.2^-1 = Id($)
Homomorphism of GrpFP into GrpPerm: $, Degree 6, Order 2 * 3 induced by
    $.1 |--> (1, 2, 4, 6, 5, 3)
    $.2 |--> Id($)
Mapping from: Abelian Group isomorphic to Z/729
Defined on 1 generator
Relations:
    729*$.1 = 0 to GrpFP
> r,mr:=RayClassGroup(3^6*MaximalOrder(L));    
> r;
Abelian Group isomorphic to Z/243
Defined on 1 generator
Relations:
    243*r.1 = 0
> A:= AbelianExtension(mr);
>   CM,w1,w2,w3 := CohomologyModule(A);
>   CohomologyGroup(CM, 2);
Full Quotient RSpace of degree 1 over Integer Ring
Column moduli:
[ 3 ]
> mm:= hom<car<G,G>->L| x:-> x@u@@f1CL@CL_L>;
> IdentifyTwoCocycle(CM,func<x | mm(x[1],x[2])@@mr@@w3>);
(1)





\end{lstlisting}








\begin{lstlisting}
K:=NumberField(x^2-5);
> o :=MaximalOrder(K);
> m :=8*o;                
> r,mr :=RayClassGroup(m,[1..2]);
> A :=AbelianExtension(mr);
> L :=NumberField(A);
> G,_,psi :=AutomorphismGroup(A);
> ar :=ArtinMap(A);
> f :=hom<r->G|x:-> x@mr@ar@Inverse(psi)>;
> Kernel(f);
Abelian Group of order 1
> IsSurjective(f);
> R,mR :=RayResidueRing(m,[1..2]);              
> f :=hom<R->G|x:-> x@mR@ar@Inverse(psi)>;
> q,mq :=quo<R|Kernel(f)>;   
> quo< r|[q.i @@mq@mR@@mr : i in [1..Ngens(q)]]> ;
Abelian Group of order 1
Mapping from: GrpAb: r to Abelian Group of order 1
\end{lstlisting}



\section{Creating Normal Extension}

\begin{lstlisting}
> K:=NumberField(PolynomialWithGaloisGroup(6,2));
> r,mr:=RayClassGroup(9*MaximalOrder(K),[1..6]);
>  A:=AbelianExtension(mr);                      
> L:=NumberField(A);
> q,mq:=quo<r|SylowSubgroup(r,3)>;                 
> A1:=AbelianExtension(Inverse(mq)*mr);
> A1;
FldAb, defined by (<[9, 0, 0, 0, 0, 0]>, [1     2       3       4       5       
6])
of structure: Z/2 + Z/2

> l:=NumberField(A1);
> AbsoluteField(l);
Number Field with defining polynomial $.1^24 - 90*$.1^22 + 2357*$.1^20 - 
    11830*$.1^18 - 41070*$.1^16 - 581010*$.1^14 + 47630645*$.1^12 - 
    509985570*$.1^10 + 5533389010*$.1^8 - 43273483590*$.1^6 + 387605577333*$.1^4
    - 1533431619050*$.1^2 + 11681842172641 over the Rational Field
> IsNormal($1);
true
> q,mq:=quo<r|SylowSubgroup(r,2)>;
> A1:=AbelianExtension(Inverse(mq)*mr);
> A1;                                            
FldAb, defined by (<[9, 0, 0, 0, 0, 0]>, [1     2       3       4       5       
6])
of structure: Z/3

> l:=NumberField(A1);
> AbsoluteField(l);
Number Field with defining polynomial $.1^18 - 9*$.1^17 + 3*$.1^16 + 174*$.1^15 
    - 357*$.1^14 - 1083*$.1^13 + 3463*$.1^12 + 2001*$.1^11 - 13218*$.1^10 + 
    3150*$.1^9 + 22479*$.1^8 - 14883*$.1^7 - 15063*$.1^6 + 16155*$.1^5 + 
    741*$.1^4 - 5073*$.1^3 + 1557*$.1^2 - 37 over the Rational Field
> IsNormal($1);
true
> r,mr:=RayClassGroup(9*MaximalOrder(K),[1..5]);// not invariant subgroup in
   infinite place
> r;
Abelian Group isomorphic to Z/6
Defined on 2 generators
Relations:
    3*r.1 = 0
    2*r.2 = 0
> time IsNormal(AbsoluteField(NumberField(AbelianExtension(mr))));             
false
Time: 0.840





\end{lstlisting}
\section{Shafarevich-Weil theorem }
Let $L/K$ be a global field extension, then Shafarevich–Weil theorem relates the fundamental class $u_{L/K}$ to an extension of Galois groups $\Gal(L,K)$.
%\begin{definition}
%Cup Product::::::----->>>>>>>>
%\end{definition}


The global fundamental class $u_{L/K}$ can also be used to define the cup product  
\[ H^{r}(G(L/K), \mathbb{Z})\xrightarrow{\sim} H^{r+2}(G(L/K), C_{L}),   \]
for $r \in \mathbb{Z}$ which is an isomorphism.

Note: For $r=-2$, we obtain a canonical isomorphism 
\[ H^{-2}(G,\mathbb{Z})\cong G(L/K)^{ab}\rightarrow C_{K}/N_{L/K}C_{L},  \]
which is an inverse of the Artin map.\\

Also for $L/K'/K$ be a tower of field extensions such that $L/K$ is Galois then the diagrams \\
\[\begin{tikzcd} [row sep=0.4em, column sep=1.5em]
H^{r}(G, \mathbb{Z})\arrow{dd}{\rest} \arrow{r}{\sim} & H^{r+2}(G, C_{L})\arrow{dd}{\rest}  &  H^{r}(G, \mathbb{Z})\arrow{r}{\sim} & H^{r+2}(G, C_{L}) \\
  &   \hspace{3.5cm}\text{and} &   & \\
H^{r}(G', \mathbb{Z})\arrow{r}{\sim} & H^{r+2}(G', C_{L})  & H^{r}(G', \mathbb{Z})\arrow{r}{\sim}\arrow{uu}{cor}  & H^{r+2}(G', C_{L})\arrow{uu}{cor}
%\arrow{from=A, to=B, Rightarrow}
\end{tikzcd}\]
are commutative where $G= \Gal(L/K)$ and $G'=\Gal(L/K')$.\\


%{\color{blue}Missing informations}
%Shafarevich–Weil theorem relates the fundamental class $u_{L/K}$ of a Galois extension of local or global fields $L/K$ to an extension of Galois groups.

Suppose $E/L/K$ is a tower of the global field extensions such that $E/L$ is abelian and $L/K$ is normal. $\Gal(E/K)$ is an extension of $\Gal(L/K)$ by the abelian group $\Gal(E/L)$ and this extension corresponds to an element of cohomology group $H^{2}(\Gal(L/K),\Gal(E/L) )$. Let $\psi : I_{L} \rightarrow \Gal(E/L)$ be the reciprocity map and $u_{E/L}\in H^{2}(\Gal(L/K), I_L)$ be the fundamental class. Shafarevich–Weil theorem states that class of the $\Gal(E/K)$ is the image of fundamental class under the homomorphism of cohomology groups induced by the reciprocity law map. \\% (Artin- Tate 2009). \\
\textbf{Note:} 
\begin{enumerate}
\item $S$ contains all archimedean primes,
\item $S$ contains all prime divisors of $n$,
\item $J_K = K^{\times}/ J_{K,S}$  {\color{red}check,3,4}
\item $S$ contains all factors of the numerator  and denominator of $a_i$.
\end{enumerate}
Condition $4$ states  that all $a_i$ are $S$-units, that is, $a_i \in K_S =K\cap J_{K,S}$.


For a number field $K$, suppose $C_K$ is the id\`ele class group of $K$. Let $H$ be the subgroup of $C_K$ such that $[C_K : H]< \infty $ then there exists a finite abelian extension $L$ of $K$ such that norm group of $C_L$ is $H$. In this case $H$ is called normic subgroup of $C_K$. Let  $\psi : C_K \rightarrow \Gal(K^{ab} / K)$ be the Artin map then each normic subgroup of $C_K$ is the inverse image of open subgroup of $\Gal(K^{ab} / K).$\\
Let $H$ be a normic subgroup of $C_K$ and it corresponds to an abelian extension $L$ over $K$. Suppose that $H\leq H_1$ then $H_1$is also normic so we have another abelian extension $L_1/K$ such that $L1\leq L$. In this case, 
\[N_{L/K}(C_L) = N_{L_{1}/K}( N_{L/L_{1}}(C_L) )  \subseteq N_{L_{1}/K}(C_{L_{1}} ) .\]

\begin{theorem}
\begin{enumerate}
\item Every abelian extension $L/K$ satisfies the reciprocity law (i.e. there is an Artin map $\psi_{L/K}$).
\item The Artin map $\psi_{L/K}$ is surjective with kernel $K^{*}N_{L/K}(J_{L})$ and hence induces an isomorphism of $C_{K}/N_{L/K}(C_L)$ onto $G(L/K)$.
\item If $M\supset L\supset K $ are the abelian extensions, then the diagram

\[\begin{tikzcd}
C_K/N_{M/K}C_M\arrow{d}{j}\arrow{r}{\psi_{M/K}} & G(M/K)\arrow{d}{\theta}\\
C_K/N_{M/K}C_M\arrow{r}{\psi_{L/K}} &
G(L/K)
\end{tikzcd}\]
commutes (where $\theta$ is the usual map and $j$ is the natural surjective map which exists because $N_{M/K}C_M \subset N_{L/K}C_L$).
\item (\textbf{Existence Theorem.}) For every open subgroup $N$ of finite index in $C_K$ there exists a unique abelian extension $L/K$ (in a fixed algebraic closure of $K$) such that $N_{L/K} C_L = N$.


 \end{enumerate}
\end{theorem}
\begin{proof}
%Main theorem on the abelian extensions(TAKAGI-ARTIN):page-172 Cassel
\cite{Cassels}, page-172
\end{proof}

{\color{red}Have to find theory from Nakayama Paper regarding this}




\textbf{Hasse Norm Theorem:}
Let  $L/K$ be a cyclic extension of number fields, then Hasse norm theorem states that if $a\in K^{\times}$ is a local norm everywhere, then it is a global norm. {\color{blue}Here, to be a global norm means to be an element $a$ of $K$ such that there is an element $b \in L$ with $\Norm_{L/K} ( b ) = a$} \\
The theorem is no longer true in general if the extension is abelian but not cyclic. Hasse gave the counterexample that $3$ is a local norm everywhere for the extension $\mathbb{Q}( \sqrt {-3},\sqrt {13} ) / \mathbb{Q}$ but is not a global norm. Serre and Tate showed that another counterexample is given by the field $\mathbb{Q}({\sqrt{13},\sqrt{17}})/ \mathbb{Q}$  where every rational square is a local norm everywhere but $5^2$ is not a global norm.\\ 
\textbf{Cassel Fr\"ohllich page-186 for norm  and page-199 for group extension:}




\textbf{to be continued...}

\end{document}
